\documentclass[12pt]{article}
\usepackage[russian]{babel}
\usepackage[utf8x]{inputenc}
\usepackage{amssymb}
\usepackage{amsmath}
\usepackage{graphicx}
\usepackage[margin=.8in]{geometry}
\usepackage[colorinlistoftodos]{todonotes}
\usepackage{listings}
\usepackage[section]{placeins}
\usepackage[T1]{fontenc}
\begin{document}

\title{Анализ былины \guillemotleft Козарин: ("Во городе да во Цернилове\ldots") \guillemotright}
\author{Андрей Валиков, ФКН, 1 курс (маг)}
\date{}
\maketitle

\section{Жанр}
Анализируемая былина является героической, так как повествование разворачивается вокруг одного персонажа, богатыря Михаила Козарина.

\section{Тип}
Былина относится к киевскому циклу, события происходят на Волыни.
\begin{center}
\textit{Во городе да во Цернилове}
\end{center}
Цернилов - деревня на территории современной Чехии. В других изложения читаем:

\begin{center}
\textit{Как из далеча было, из Галичья, \\
Из Волынца города, из Галичья,}
\end{center}
Впоследствии Михаил приезжает к киевскому князю:

\begin{center}
\textit{Едет Михайло в славный Киев-град, \\
К ласковому князю Владимиру,}
\end{center}





\section{Образ богатыря и главная мысль}
Из-за проклятия ведьмы от героя отрекается вся семья, но не сестра, которая долгое время растит и воспитывает брата. Повзрослев, Михаил узнаёт свою историю и

Узнав от ворона о пленнице, Михаил, как и полагается богатырю, решает спасти её. Впоследствии в пленнице он узнает, сестру

Устное творчество предполагает множество различных вариантов одной истории. В анализируемой былине рассказвается, что Михаил не пошёл домой, хотя и сестра звала его. В других изложениях этой истории видим более подробное описание прибытия Михаила. Несмотря на спасённую дочь, родители не признают сына. «\ldotsна Михайлушка батюшка оцьми не звел». Нанесённая обида не позволяет ему сделать первый шаг для примирения с отвергнувшими его родителями. «Не в первый раз зашел, да в последний к вам\ldots» Богатырь Михаил Козарин сохраняет честь и достоинство, совершает подвиги не за одобрение, но потому что русский герой не может поступить по-другому.

Тем не менее, существуют редкие варианты с положительным концом, где отец король, примиряется с сыном и отдаёт ему царство. 

\section{Художественные приёмы}

Имеют место повторения, как в случае описания обращение «тотар» к пленнице, три раза повторяется одна и та же семантическая конструкция.

\end{document}
