\documentclass[12pt]{article}
\usepackage[russian]{babel}
\usepackage[utf8x]{inputenc}
\usepackage{amssymb}
\usepackage{amsmath}
\usepackage{graphicx}
\usepackage[margin=.8in]{geometry}
\usepackage[colorinlistoftodos]{todonotes}
\usepackage{listings}
\usepackage[section]{placeins}
\usepackage[T1]{fontenc}
\usepackage{verse}
\usepackage{geometry}

\setlength\parindent{0pt}
\geometry{left=1cm, right=1cm, top=1cm, bottom=1.5cm}
\begin{document}

\title{Анализ былины \guillemotleft Козарин: ("Во городе да во Цернилове\ldots") \guillemotright}
\author{Андрей Валиков, ФКН, 1 курс (маг)}
\date{}
\maketitle

\section{Жанр}
Анализируемая былина является героической, так как повествование разворачивается вокруг одного персонажа, богатыря Михаила Козарина.

\section{Тип}
Былина относится к киевскому циклу, события происходят на Волыни.

\settowidth{\versewidth}{Богатырскому серцу розгоретися,}
\begin{verse}[\versewidth]
\itshape
Во городе да во Цернилове
\end{verse}

Цернилов - деревня на территории современной Чехии. В других изложения читаем:

\settowidth{\versewidth}{Богатырскому серцу розгоретися,}
\begin{verse}[\versewidth]
\itshape
Как из далеча было, из Галичья, \\
Из Волынца города, из Галичья,
\end{verse}

А также на киевской земле.
\settowidth{\versewidth}{Богатырскому серцу розгоретися,}
\begin{verse}[\versewidth]
\itshape
Едет Михайло в славный Киев-град, \\
К ласковому князю Владимиру,
\end{verse}


\section{Анализ персонажей}

\subsection{Михаил Козарин}
Главный герой былины.

\subsection{Пётр Коромыслов}
Отец Михаила, отказавшийся от него из-за проклятия ведьмы.

\subsection{Марфа}
Сестра Михаила, единственная из семьи, которая не испугалась проклятия ведьмы и взялась заботиться о нём. Далее окажется пленённой разбойниками.

\subsection{Чёрный ворон}
Птица, встретившаяся Михаилу в лесу (иногда говорится, что в это время Михаил был на охоте, поэтому пощада ворону приобретает дополнительный смысл). Указывает герою, где искать пленнцу.

\subsection{Разбойники}
Бандиты, пленившие сестру Михаила.



\section{Образ богатыря и главная мысль}

Из-за проклятия ведьмы от героя отрекается вся семья, но не сестра, которая долгое время растит и воспитывает брата. Повзрослев, Михаил узнаёт свою историю и забирает у отца богатырское снаряжение с тем, чтобы отправиться на службу к киевскому князю.
Князь поручает отправиться на охоту, чтобы настрелять птиц к столу. После охоты Михаил встречает ворона, которого поначалу решает убить, но выслушав историю о том что на горе разбойники держат в плену девушку ворона о пленнице, Михаил, как и полагается богатырю, решает спасти её, а ворона пощадить. «Не честь-хвала богатырю убить чёрна ворона, А честь-хвала вернуть русскую пленницу.» Впоследствии в пленнице он узнает, сестру, которую решает отвезти домой.

Устное творчество предполагает множество различных вариантов одной истории. В анализируемой былине рассказвается, что Михаил не пошёл домой, хотя и сестра звала его. В других изложениях этой истории видим более подробное описание прибытия Михаила. Несмотря на спасённую дочь, родители не признают сына. «\ldotsна Михайлушка батюшка оцьми не звел». Нанесённая обида не позволяет ему сделать первый шаг для примирения с отвергнувшими его родителями. «Не в первый раз зашел, да в последний к вам\ldots» Богатырь Михаил Козарин сохраняет честь и достоинство, совершает подвиги не за одобрение, но потому что русский герой не может поступить по-другому.

Тем не менее, существуют редкие варианты с положительным концом, где отец король, примиряется с сыном и отдаёт ему царство.

Таким образом, несмотря на то, как несправедливо обошлась семья с Михаилом, он является героем-спасителем, возможно именно благодаря заботе подаренной ему его сестрой. Тем не менее, обида, нанесённая семьёй столь глубока, что не позволяет Михаилу искать пути для примирения с ними, поэтому когда отец, несмотря на спасение дочери, продолжает ненавидеть сына, Михаил уходит не прося ни о чём.


\section{Художественные приёмы}

\subsection{Эпитеты}

\begin{itemize}

\item \ldotsдевицю да душу красную

\item \ldotsего да лошадь добрая,

\item трубцяту косу

\item буйну голову

\end{itemize}

\subsection{Слова с уменьшительно-ласкательными суффиксами}

\begin{itemize}

\item цядышко
\item малешенько
\item глупешенько
\item братёлко
\item вороновиця

\end{itemize}

\subsection{Повторы}

\subsubsection{Снаряжение Михаила}


\settowidth{\versewidth}{Богатырскому серцу розгоретися,}
\begin{verse}[\versewidth]
\itshape
Выходил Козарин на конюшон двор, \\
Выбирал Козарин лошадь добрую,
\end{verse}

\subsubsection{Сцена с вороном}

\settowidth{\versewidth}{Богатырскому серцу розгоретися,}
\begin{verse}[\versewidth]
\itshape
Да хотел Козарин его конём стоптать \\
Да конём стоптать, его жезлом сколоть.
\end{verse}


\subsubsection{Описание разбойников}

\settowidth{\versewidth}{Богатырскому серцу розгоретися,}
\begin{verse}[\versewidth]
\itshape
Во цистом-то поли три шатра стоит, \\
Три шатра стоит белополотняных; \\
Во шатрах есь да три тотарина, \\
Три тотарина да три углановья*;
\end{verse}



\subsubsection{Описание обращения разбойников к девушке \\(2 раза и один уменьшенный стих)}


\settowidth{\versewidth}{Богатырскому серцу розгоретися,}
\begin{verse}[\versewidth]
\itshape
Как один-от тотарин девушку уговариват: \\
«Ты не плаць, девиця да душа красная:\\
Я возьму тобя да за больша сына, — \\
Ишше будёш у меня да большой клюцницей, \\
Большей клюцьницей, большей замоцьницей!..» \\
Ешше плацет девиця пуще старого, \\
Пуще старого да пуще прежного. \\
Ишше цёшот девиця да буйну голову, \\
Заплетаёт девиця да трубцяту косу, \\
Она сама косы да приговарыват: \\
«Ты коса-ль моя да коса русая! \\
Некому (так), коса моя, досталасе: \\
Не кнезьям, коса, да не бояринам, — \\
Ты досталась, коса, да трём тотаринам, \\
Трём тотаринам да трём поганыем!»
\end{verse}



\end{document}
